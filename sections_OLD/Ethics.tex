
\chapter{Ethics and societal aspects}
%Samhälleliga och etiska aspekter, bedömning om det behöver beaktas för vald problemställning I planeringsrapporten förväntas gruppen skriva en kortare text där gruppen bedömer om samhälleliga och etiska aspekter behöver beaktas och analyseras vidare i uppsatsen/rapporten. Gruppen använder sig med fördel av bilaga 7 som stöd samt de digitala resurser som finns på Studentportalens sidor om kandidatarbetet.
\section{Legality of emulators vs. ROMs}
Most digital copyrighted things have some sort of controversy around them. Gaming consoles and their respective emulators are no different. Like most subjects touching on potential piracy and copyright infringement, there is a huge grey zone whether or not something is legal or illegal, and subsequently if something is right or wrong. 

\subsection{The emulator}
The creation of a game emulator is generally not illegal as long as no proprietary code is stolen from the original console\cite{emulatorLegal}. This is mainly because the emulator itself is just a remade version made to replicate the original console. Some consoles do have proprietary code in their BIOS though, which might cause a problem for this project when emulating the Boot ROM. 

\subsection{The ROMs}
The ROMs themselves are in a bit of a grey zone\cite{romLegal}. Making back ups of game cartridges one already owns is completely legal under the right circumstances and for the right purposes according to US copyright laws\cite{section117}. Selling or distributing said copies is illegal and counts as copyright infringement, and so does downloading other people's copies. The main legal way to get a playable version of an existing game's ROM seems to be through making your own copy of that specific game, which you must already own. Piracy and copyright laws do vary from country to country though, and there is continuous debate online whether or not these laws are for the greater good or not\cite{emulatorPodcast}.

\subsection{Test-ROMs}
%On the topic of of developing an emulator which has the ability to emulate copyright-infringing ROMs the question regarding the acquisition of legal game ROMs which are usable for testing.

If game ROMs are not acquirable for some reason, there are a multitude of test-ROMs online that are available to make sure the emulator and its parts work as intended\cite{testROMs}. These are free to use and will benefit the troubleshooting of this project immensely without any risk for potential copyright infringement.


\section{Sharing knowledge}
By creating an emulator for an old console one gets to thoroughly understand how the very basic components of a gaming system work together to create a complete console. By making the emulator's source code available publicly, it might help increase the knowledge for these types of systems among students, enthusiasts and developers alike. As long as no proprietary code is shared, there is no obvious way this could harm anyone. If the Game Boy was not discontinued then one could argue that an emulator would deter people from buying an expensive console. However, since the Game Boy is no longer being sold, that risk is effectively eliminated. On the contrary, finding an emulator might even contribute to further interest in Nintendo products which would most likely benefit them.


\section{Gaming history preservation}
One of the main pro-emulation arguments is that of history preservation. The Game Boy was made with hardware that degrades over time, which limits each units lifetime. Ever since the product was discontinued in 2003\cite{gameBoyDisc} people have been urging to save what many believes to be one of the greatest gaming products ever released. Much like museums would save objects from historical events, Game Boy emulators and ROM archiving could be seen as a kind of digital museum where the soon to be lost hardware is forever stored. Several games for the Game Boy are no longer being sold, and acquiring a used copy becomes increasingly more difficult with time. This makes retro game emulation for historic preservation quite an attractive option for a lot of people. However the legal situation around emulators is very much a grey zone, which makes justifying an emulator for this purpose much more difficult. Creating an emulator would definitely contribute to increased history preservation, however through incorrect use it could also be a tool for playing illegal copies of games.

\subsection{Console Classix}
An example of a company who has tried to make old ROMs playable to the public is Console Classix\cite{romLegal}. The idea is to rent their games through a client-server solution to make the games playable on a home PC. Although they have received a letter from Nintendo regarding their business\cite{letterFromNintendo}, no legal action has been taken since. Console Classix's defence is that they, unlike illegal ROM sharing websites, do not publish the ROMs publicly but rather provide limited access to them with a subscription method. By sending the ROM images directly from the server to the client's RAM, the game effectively disappears from the client when they stop playing, which would prevent any permanent distributing from happening. Additionally, they do not rent more copies out than they own, therefore it is difficult to build a case around copyright infringement as well. This is a great example of using an emulator seemingly legally to still fill the purpose of history preservation, while simultaneously providing a way to play retro games for those subscribed to their service.


%\begin{itemize}
 %   \item Nintendo och copyright
  %  \item Piracy/Legal?
   %     Different countries, different rules. - Format shifting for personal backups (Old VHS to DVD comparison?)
    %    Downloading vs making your own copies
     %   
    %\item Emulator vs ROM? Vilket är lagligt/olagligt?
     %   Emulator legal
      %  Rom questionable Greyzone
%\end{itemize}
%\\\\
%intressant länk jag hittade när jag kollade annat https://www.tomshardware.com/news/why-most-roms-are-illegal,37512.html mvh isakfisak tack mvh kvickedick




