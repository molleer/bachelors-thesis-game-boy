\chapter{Timeplan}
%Den här delen av planeringsrapporten beskriver vad som ska göras och när det ska göras. Personer som ska kontaktas bör också stå med här. Datum eller åtminstone veckor då studenterna ska ge delrapporter samt slutgiltiga presentationen ska stå här. Tidsplanen kommer naturligtvis vara rätt grov i början.

%Det är viktigt att notera att aktiviteterna inom projektet inte kan ske sekventiellt då dessa aktiviteter är beroende av varandra, vilket innebär att ett antal iterationer mellan dem kommer att ske. Endast genom att iterera mellan dem kommer den uppbyggda kunskapen bli utnyttjad på ett bra sätt. Samma tänkande gäller också rapportskrivandet, det vill säga uppdatering av ett avsnitt kräver att även andra uppdateras. Rapportskrivande ska därför ske kontinuerligt under hela projektet.

%\begin{enumerate}
%    \item Se project plan
%\end{enumerate}
In addition to the plan shown below, the group will also continuously work on writing the report, gathering sources, researching new material and further investigate the hardware of the Game Boy. The group is also applying a type of agile working process, called Scrum\cite{Scrum}, where the work needed do be done is re-evaluated continuously and re-prioritised, making the plan shown below subject to change.
\begin{table}[]
\begin{tabular}{|l|l|}
\hline
Week & Plan                                                                                            \\ \hline
3    & \begin{tabular}[c]{@{}l@{}}Meeting with supervisor and group\\ Group contract.\end{tabular}     \\ \hline
4    & Research GameBoy hardware                                                                       \\ \hline
5  & \begin{tabular}[c]{@{}l@{}}Research GameBoy hardware\\ Make high level UML and plan implementation\end{tabular}                              \\ \hline
6  & \begin{tabular}[c]{@{}l@{}}Implementation of basic MMU and CPU \\ \textbf{Project Plan}\\ Travis set up on GitHub\\ OpenGL project set up\end{tabular} \\ \hline
7    & Display Nintendo Logo (Basic PPU,CPU and MMU working)                                           \\ \hline
8    & Continued development of CPU,PPU and MMU                                                        \\ \hline
9    & Basic IO                                                                                        \\ \hline
10   & \textbf{\begin{tabular}[c]{@{}l@{}}Oral halftime presentation\\ Self evaluation 1\end{tabular}} \\ \hline
11   & \textbf{EXAM WEEK}                                                                              \\ \hline
12   & Preparations for first playable ROM                                                             \\ \hline
13   & Non-proprietary ROM bootable/playable                                                                        \\ \hline
14   & Use of Test-ROMs to further develop each part of the Game Boy                                    \\ \hline
15   & Final touches on development of Game Boy                                                        \\ \hline
16   & Report work                                                                                     \\ \hline
17   & \textbf{First review of final report}                                                           \\ \hline
18   & MVP Finished                                                                                    \\ \hline
19 & \textbf{\begin{tabular}[c]{@{}l@{}}Finished report with English title\\ Contribution report\end{tabular}}                                    \\ \hline
20   & \textbf{\begin{tabular}[c]{@{}l@{}}Film hand-in\\ Written individual opposition\end{tabular}}   \\ \hline
21 & \textbf{\begin{tabular}[c]{@{}l@{}}Virtual exhibition\\ Final presentation\\ Self evaluation 2\end{tabular}}                                 \\ \hline
22   & \textbf{\begin{tabular}[c]{@{}l@{}}Final hand-in\\ Finished report\end{tabular}}                \\ \hline
\end{tabular}
\end{table}
