\chapter{Limitations}\label{Limitations}
% Avgränsningarna ska ta upp vilka delar av problemet som inte tas upp i uppsatsen/rapporten, och anledningen till detta. Motivering av avgränsningarna är viktigt.


Developing an emulator is a big task. 
There are practically no end to features that could be added.
Therefore some kind of limitations are needed to be set, such that the project can be finished.

\section{Limit emulation to the original Game Boy}
There are many different versions of the Game Boy which all happens to be backwards compatible\cite{compatibility}.
The first limitation to the project is to only support emulation of games made for the original Game Boy.
This is mostly because it will reduce the scope of the project to fit the time frame.
Although, due to the backwards compatibility the project can also later be further improved to support additional newer versions of the Game Boy.


\section{Level of emulation}
Emulating hardware can be done in different depth.
For example a system could be emulated to the extent that all electric wires are emulated.
For this project that approach is overkill.
Instead the level of emulation that will be used is to emulate only the functionality of some of the internal components and the system as a whole.
To be able to run a game, which is one of the goals of the project, there is no need to emulate at a lower level than just to the functionality of the system.
\\\\
To make the project more modular and easy to work with some of the major internal components of the Game Boy will be emulated with regards to their functions.
Putting these together will result in a emulated system.


\section{Cartridge support}
The cartridge slot on the Game Boy is most common for holding the game cartridge, called Game Pak, but there are also many different accessories that were released.
The most notable of accessories are the Game Boy Camera and the Game Boy Printer.
With this said there are a lot of different hardware that could be supported.
This project will be limited to supporting Game Paks, but even the Game Paks contain different variations of hardware.
Game Paks can have different Memory Bank Controllers (MBC) and might also include external RAM etc \cite{cartridgeType}.
Although, to play the simplest of games there is no need to emulate a MBC, because there are none in those Game Paks and this project will therefore not implement these extra Game Pack hardware.
There should be no issue to support more cartridge types down the line if there is time for that.

\section{Sound}
To interact with a game on the Game Boy there are three different major components that makes the experience richer. 
One of the components that is important is user input.
This is handled via all the buttons on the Game Boy.da
The second component is the graphics which is one of the ways the Game Boy convey information to the user.
The last component, which also is the other way to convey information from the game to the user, is the sound.
Out of these three components there is one that is less crucial to the ability to play a game. 
That component is the sound.
\\\\
If you remove the screen the user have no information about your position in the game and is clueless if your actions make progress for you.
If you instead remove the buttons the user have no way to control the game or character.
But, if you remove the sound from a game the user often manage to still play the game.
The level of enjoyment might not be as high as if you have all of the three components but that clearly shows what part of the development that has to prioritised.
So, in this project the sound will not be prioritised but implemented if there is time over to do so.

\section{The Minimum Viable Product}
The Minimum Viable Product (MVP) is the minimum goal set up for providing a solution to the problem. 
The chosen MVP consists of having a bootable Game Boy emulator which can fully play the simple games, but without sounds.
