\chapter{Background}
%Bakgrund ska innehålla en motivering till varför det valda ämnet är intressant ur akademisk synvinkel och/eller ur tekniskt perspektiv eller i förekommande fall ur kundens/uppdragsgivarens perspektiv. I vissa fall ska den här rubriken inkludera en kort historik över ämnet. Efter att ha läst bakgrunden ska alla läsare förstå varför ämnet är relevant. Följande frågeställningar bör vara aktuella:
%Vad är ämnet/problemet som ska undersökas? Varför har ämnet/problemet uppkommit? Varför och för vem är det ett relevant eller intressant ämne/problem? Kan det specifika ämnet/problemet relateras till en mer generell diskussion?

%Från kandidat-beskrivningen

%Background
%The goal of this project is to build an application that can emulate the hardware for an old-school console system such as the Game Boy handheld console and allow the user to play games that were built for it.

\section{What is an emulator?}
An emulator may refer to some kind of hardware or a piece of software which emulates the functionality of some other computer system\cite{emulatortechnopedia}\cite{emulatorlifewire}. While the main goal of an emulator is to imitate another computer system, an emulator may also improve upon the original system by providing additional features or by exceeding the performance of the original system\cite{emulatorlifewire}.

\section{A brief history of Game Boy emulation}
The Game Boy is a handheld video game console by Nintendo released for the first time in 1989\cite{gameboy}. It is characterised by its green monochromatic LCD display and grey and bulky design (see figure \ref{fig:GameBoy-fig})\cite{gameboylook}. 

\begin{figure}[H]
    \centering
    \includegraphics[scale=0.7]{figures/GameBoy.jpg}
    \caption{The original game boy. Photo: Wiliam Warby (CC BY 2.0)}
    \label{fig:GameBoy-fig}
\end{figure}

The first known emulator of the Game Boy released, that could run commercial games, was an emulator by the name Virtual Game Boy. It was created by Marat Fayzullin in 1995 for some unknown system and was ported to PC somewhere between 1995 and 1996\cite{gameboyemulationhistory}. Since then many more Game Boy emulators have been released; written in many different languages and ranging in functionality\cite{gameboyemulators}.

\section{Why is this subject interesting?}
While creating a Game Boy emulator may not be interesting in the sense that it pushes academic boundaries, it is interesting from a learning point of view. There are many technical aspects which a software developer normally does not have to care about, but which need to be considered when creating an emulator. The project will be well documented and is in a public repository. In the end our contribution will make the topic of Game Boy emulation more accessible for others interested in the topic.

\section{Terminology}

\begin{itemize}
    \item ROM - A read only memory. A memory which in the context of emulator development contains the game data.
    \item MVP - Minimum Viable Product, is within Lean Startup considered a product which allows a team to collect maximum amount of validated learning about customers with the least effort\cite{Lean}. Within this context another definition could be a bare bones product which still could be considered a complete product. 
    \item CI - Continuous Integration is the process of build automation, where a projects builds or tests are automatically run upon integration\cite{CI}.
    \item RAM - A random access memory. The data which is contained in the RAM is volatile - it can only store data as long as it has power.
    
    %considered more as the minimum specification for what is considered a 
\end{itemize}