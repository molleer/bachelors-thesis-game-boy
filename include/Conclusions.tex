\chapter{Conclusions}
Emulating an older system, such as the Game Boy, can be done to some accuracy with limited resources, both in time and available knowledge, as shown by this thesis. Despite not perfectly replicating the system to be emulated, a Game Boy emulator can be created which runs many games and provides the functionality expected from an actual Game Boy, thereby both providing a more or less authentic experience as well as a way to preserve games. It does however mean that the emulator can exhibit strange behaviour, freeze, or even crash when unexpected problems regarding the emulation are encountered and can therefore not be considered stable. If better, more detailed, and comprehensible documentation could be provided, better emulators could also be produced. One could even consider that there is something of a finish line for emulator development, where if you reproduce the exact behaviour of the hardware, the topic is exhausted. If this would be achieved, one could look at other purposes of emulator development such as using it as a tool for education. This could for example be done by creating a simple, well-documented emulator with a clear structure showing how the translation from hardware to software is done, leveraging the interest for gaming in the general population to work as an inspiration to learn more about low-level programming.