
\thispagestyle{plain}			% Supress header 
\setlength{\parskip}{0pt plus 1.0pt}
\section*{\centering Abstract}
% Abstract text about your project. The following text should not appear in your report.
This thesis studies the subject of system emulation through the development of a set of software microcontrollers and the assembling of them into a complex system. The specific system aimed to be emulated is the original Game Boy released in 1989. This requires the developers to reproduce specific hardware behaviour through software and therefore requires certain knowledge of the system which is to be emulated. While the Game Boy is a proprietary product owned by Nintendo, the produced system uses no copyrighted material. %Furthermore, it shows that an emulator in fact can be created by combining a set of microcontrollers.
\\\\
Through the use of documentation provided by the reverse engineering of the original hardware done by members of the community, this thesis shows that an emulator can be created by combining a set of software microcontrollers. Moreover, it is concluded that while the academic interest in the emulation of simple systems might be limited, it could also could be used to generate interest in low-level programming.
%The thesis shows that an emulator in fact can be created by combining a set of microcontrollers. It also shows that it is possible through the use of documentation provided by the reverse engineering of the original hardware done by members of the emulating community. Furthermore we conclude that while the academic interest in system emulation might be limited, it could be used to generate interest in low-level programming.

%Shown in this thesis is the fact that an emulator can be created by a combining a set of microcontrollers. Additionally, it also shows that it is possible through the use of documentation provided by the reverse engineering of the original hardware done by members of the emulating community. Furthermore it is concluded that while the academic interest in system emulation might be limited, it could be used to generate interest in low-level programming.
% It also shows that it is possible through the use of documentation provided by the reverse engineering of the original hardware done by members of the emulating community. Furthermore, it is concluded that while the academic interest in system emulation might be limited, it could be used to generate interest in low-level programming.

%This thesis studies the subject of system emulation through emulation of a set of microcontrollers and assembling them into a complex system. The specific system aimed to be emulated is the original Game Boy released in 1989. This requires the developers to emulate specific hardware behaviour through software and therefore requires certain knowledge of the system which is to be emulated. While the Game Boy is a proprietary product owned by Nintendo, the emulator created uses no copyrighted material. %Furthermore, it shows that an emulator in fact can be created by combining a set of microcontrollers.



% KEYWORDS (MAXIMUM 10 WORDS)
\vfill
Keywords: Emulation, Gameboy, Game Boy, C++, OpenGL, OpenAL, ImGui

\newpage				% Create empty back of side
\thispagestyle{plain}
\mbox{}


