\thispagestyle{plain}			% Supress header 
\setlength{\parskip}{0pt plus 1.0pt}
\section*{\centering Sammandrag}
Denna kandidatuppsats studerar systememulation genom att utveckla ett flertal mjukvarumikrokontroller som därefter kombineras för att tillsammans bilda ett komplext system. Det specifika systemet som ämnas emuleras är den första Game Boy-konsollen, släppt 1989. För att genomföra detta krävs att utvecklarna reproducerar den specifika hårdvaran i mjukvara, vilket kräver viss kunskap om det ursprungliga systemet. 
Då Game Boy är en licensierad produkt, ägd av Nintendo, är det värt att notera att det framtagna systemet ej använder något upphovsrättsskyddat material.
%Game Boy är en licensierad produkt, ägd av Nintendo, av vilken anledning emulatorn som utvecklats inte använder sig av något upphovsrättsskyddad material. 
\\\\
Dokumentationen som använts för att skapa denna emulator har tagits fram genom att demontera och undersöka originalhårdvaran. Detta har gjorts av ett flertal människor som är del av en emulatorintresserad internetgemenskap. Avslutningsvis konstateras att det, trots det begränsade akademiska intresset för systememulering av enkla system, finns potential för att använda sig av det för att generera intresse för lågnivåprogrammering.


%, varefter det akademiska intresset för systememulering av enkla system må vara begränsat, det finns potential för att använda sig av det för att generera intresse för lågnivåprogrammering 



%Genom att demontera och undersöka originalhårdvaran har  att använda dokumentation som producerats av människor intresserade av emulering som 

%This thesis studies the subject of system emulation through emulation of a set of microcontrollers and assembling them into a complex system. The specific system aimed to be emulated is the original Game Boy released in 1989. This requires the developers to emulate specific hardware behaviour through software and therefore requires certain knowledge of the system which is to be emulated. While the Game Boy is a proprietary product owned by Nintendo, the emulator created uses no copyrighted material. Furthermore, it shows that an emulator in fact can be created by combining a set of microcontrollers.
%\\\\
%Through the use of documentation provided by the reverse engineering of the original hardware done by members of the community, this thesis shows that an emulator in fact can be created by combining a set of microcontrollers. Moreover, it is concluded that while the academic interest in system emulation might be limited, it could be used to generate interest in low-level programming.

% KEYWORDS (MAXIMUM 10 WORDS)
\vfill
Nyckelord: Emulering, Gameboy, Game Boy, C++, OpenGL, OpenAL, ImGui

\newpage				% Create empty back of side
\thispagestyle{plain}
\mbox{}